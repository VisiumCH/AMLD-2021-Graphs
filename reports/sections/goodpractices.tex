
\subsection{Good \LaTeX  practices}
\label{subsec:practices}
<Don't forget to label your Figures, Tables, Sections, etc... to easily create H-ref links in the PDF!>

<\GR{You can easily add visible comments in the PDF by defining your personal command}>

<Make sure each main Section start on a newpage with \textbackslash newpage >

<Include your references in the .bib and cite them with the  \textbackslash cite command \cite{goodfellow2014generative} >
\newpage
\subsection{Figures}
\label{subsec:figures}

<Add in Figures like this:>
\begin{figure}[H]
  \centering
  \includegraphics[width=\columnwidth]{horizontal-color.png}
  \caption{Example of Figure}
  \vspace{-3mm}
  \label{fig:fig_example}
\end{figure}
< and reference them properly using the hyperref package : Figure \ref{fig:fig_example}>

\subsubsection{Multiple figures}
\label{subsub:multiple}
<Include Figures with multiple image like this: 
\begin{figure}[H]
\centering
  \begin{subfigure}[t]{0.45\textwidth}
    \includegraphics[width=\textwidth]{horizontal-color.png}  
    \caption{\nth{1} caption}
    \label{subfig:sub1}
  \end{subfigure}
  \hfill
  \begin{subfigure}[t]{0.45\textwidth}
    \includegraphics[width=\textwidth]{horizontal-color.png} 
    \caption{\nth{2} caption}
    \label{subfig:sub2}
  \end{subfigure}
\caption{Big caption}    
\label{fig:multifigs}
\end{figure}  

<And also reference them: Figure \ref{subfig:sub2}>

\subsection{Tables}
\label{subsec:tables}
You can also use Tables:

\begin{table}[H]
    \centering
    \begin{tabular}{@{}lllll@{}}
    \toprule
    \multicolumn{5}{c}{This is a Table} \\ \midrule
    1     & 2     & 3    & 4    & 5     \\
    a     & b     & c    & d    & e     \\
    +     & -     & *    & /    & \%    \\ \bottomrule
    \end{tabular}
    \caption{Example of Table}
    \label{tab:table_example}
\end{table}

For easy design of tables, go to \url{https://www.tablesgenerator.com/}. Try to always use the Booktabs mode!

\subsection{Code snippets}
\label{subsec:code}

You can include code snippet using the minted (\url{https://www.overleaf.com/learn/latex/Code_Highlighting_with_minted}) package. It provides integration of multiple languages. For example:

\begin{minted}{python}
import numpy as np
 
def incmatrix(genl1,genl2):
    m = len(genl1)
    n = len(genl2)
    M = None #to become the incidence matrix
    VT = np.zeros((n*m,1), int)  #dummy variable
 
    #compute the bitwise xor matrix
    M1 = bitxormatrix(genl1)
    M2 = np.triu(bitxormatrix(genl2),1) 
...
\end{minted}
